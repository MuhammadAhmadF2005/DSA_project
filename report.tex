\documentclass[journal]{IEEEtran}

% -------------------- Packages --------------------
\usepackage[utf8]{inputenc}
\usepackage{graphicx}
\usepackage{amsmath, amssymb}
\usepackage{algorithm}
\usepackage{algpseudocode}
\usepackage{listings}
\usepackage{xcolor}
\usepackage{hyperref}
\usepackage{booktabs}
\usepackage{multirow}
\usepackage{array}
\usepackage{float}

\hypersetup{
    colorlinks=true,
    linkcolor=black,
    citecolor=black,
    urlcolor=blue
}

% -------------------- Code Listing Style --------------------
\lstset{
    language=C++,
    basicstyle=\ttfamily\small,
    keywordstyle=\color{blue},
    stringstyle=\color{red},
    commentstyle=\color{green!50!black},
    numbers=left,
    numberstyle=\tiny\color{gray},
    numbersep=5pt,
    frame=single,
    breaklines=true,
    tabsize=2,
    showstringspaces=false
}

\begin{document}

% ==================== COVER PAGE ====================
\begin{titlepage}
\centering
\vspace*{1cm}

\includegraphics[width=0.35\textwidth]{giki-logo.png}\\[1cm]

{\Large \textbf{Ghulam Ishaq Khan Institute of Engineering Sciences and Technology}}\\[0.4cm]
{\large Department of Computer Science}\\[1.2cm]

{\Huge \textbf{Intelligent Transport Network Management System (ITNMS)}}\\[0.4cm]
{\Large A Comprehensive Application of Data Structures and Algorithms}\\[1.5cm]

\textbf{Submitted By}\\[0.3cm]
Muhammad Ahmad - 2024335\\
Muhammad Haider - 2024385\\
Muhammad Zaid - 2024491\\
Raja Hamza Sikandar - 2024532\\

\vspace{1cm}
\textbf{Course}\\
CS221 -- Data Structures \& Algorithms\\[0.8cm]

\textbf{Semester}\\
Fall 2025

\vfill
\end{titlepage}

\clearpage

% ==================== ABSTRACT ====================
\begin{abstract}
This paper presents the design and implementation of the Intelligent Transport Network Management System (ITNMS), a comprehensive software system that simulates a smart city transportation network. The project demonstrates the practical application of fundamental data structures and algorithms including graphs, hash tables, queues, stacks, trees, and heaps. Core functionalities include route management, passenger ticketing, vehicle database operations, and analytical reporting. All data structures and algorithms are implemented manually without relying on C++ Standard Template Library (STL) containers, ensuring a strong conceptual understanding of core computer science principles. The system illustrates how theoretical concepts in data structures and algorithms can be effectively applied to real-world transportation management problems.
\end{abstract}

\begin{IEEEkeywords}
Data Structures, Algorithms, Graph Theory, Hash Tables, Transportation Systems, Dijkstra’s Algorithm, Minimum Spanning Tree
\end{IEEEkeywords}

% ==================== INTRODUCTION ====================
\section{Introduction}

The Intelligent Transport Network Management System (ITNMS) is a semester project developed for the CS221 Data Structures and Algorithms course. The project provides hands-on experience in applying fundamental data structures and algorithms to the design of a realistic transportation management system.

Modern transportation infrastructures require efficient management of routes, stations, vehicles, and passengers. Similar to metro systems, bus routing platforms, and navigation services, ITNMS models a city-wide transport network. The system integrates nearly all major concepts covered in CS221, including arrays, linked lists, stacks, queues, trees, hashing, graphs, and searching and sorting algorithms.

The primary objectives of this project are:
\begin{itemize}
    \item Demonstrate practical application of core data structures and algorithms.
    \item Implement a modular, extensible system using object-oriented principles.
    \item Analyze time and space complexity of implemented algorithms.
    \item Bridge the gap between theoretical concepts and real-world system design.
\end{itemize}

% ==================== SYSTEM REQUIREMENTS ====================
\section{System Requirement Specifications (SRS)}

\subsection{Functional Requirements}

\subsubsection{Route and Station Management}
\begin{itemize}
    \item Add and remove stations in the transport network.
    \item Add and remove routes connecting stations.
    \item Display connected stations.
    \item Perform Breadth-First Search (BFS) and Depth-First Search (DFS).
    \item Compute shortest paths using Dijkstra’s algorithm.
    \item Generate Minimum Spanning Tree (MST) using Kruskal's.
    \item Detect cycles in the graph.
\end{itemize}

\subsubsection{Passenger Ticketing System}
\begin{itemize}
    \item FIFO queue implementation for ticket requests.
    \item Enqueue and dequeue passengers.
    \item Display waiting queue.
\end{itemize}

\subsubsection{Vehicle Database Management}
\begin{itemize}
    \item Insert, search, and delete vehicles.
    \item Collision handling using chaining.
\end{itemize}

\subsubsection{History and Undo Operations}
\begin{itemize}
    \item Maintain action history using stack.
    \item Undo last performed operation.
\end{itemize}

\subsubsection{Searching and Sorting Module}
\begin{itemize}
    \item Linear and binary search algorithms.
    \item Bubble, selection, insertion, merge, quick, and heap sort algorithms.
\end{itemize}

% ==================== SYSTEM DESIGN ====================
\section{System Design}

\subsection{Architecture Overview}

The system follows a modular, object-oriented architecture where each major functionality is encapsulated within separate classes. The design is divided into data models (entities) and core engine classes.

\subsection{Class Diagram Description}

The system architecture is visualized through a comprehensive class diagram that defines the following relationships:

\begin{itemize}
    \item \textbf{Data Models:} The classes \texttt{Station}, \texttt{Route}, \texttt{Passenger}, \texttt{Vehicle}, and \texttt{LogEntry} serve as the primary entities. Each utilizes self-referential pointers (\texttt{next}) to facilitate manual linked-list management.
    \item \textbf{Core Data Structures:} 
    \begin{itemize}
        \item \texttt{CityGraph} acts as the backbone, managing an array of \texttt{Station} pointers and an adjacency list of \texttt{Route} objects.
        \item \texttt{VehicleHashTable} implements a fixed-size array of \texttt{Vehicle} pointers for $O(1)$ average-case access.
        \item \texttt{PassengerQueue} and \texttt{HistoryStack} manage the dynamic flow of passengers and system logs respectively.
    \end{itemize}
    \item \textbf{Advanced Modules:} The \texttt{Analytics} class represents a higher-level logic layer that depends on the \texttt{CityGraph}, \texttt{MinHeap}, and \texttt{BST} to generate insights such as traffic density and busiest routes.
    \item \textbf{Utility Classes:} The \texttt{SortSearchUtils} class provides a suite of static-like methods that return \texttt{ComplexityMetrics} objects, allowing the system to track performance data in real-time.
\end{itemize}

\begin{figure}[!t]
    \centering
    \includegraphics[width=0.9\linewidth]{class-diagram.png}
    \caption{UML Class Diagram for the Intelligent Transport Network Management System}
    \label{fig:class_diagram}
\end{figure}


\subsection{Graph Representation}

The transportation network is represented as a weighted, undirected graph using an adjacency list representation. Stations correspond to vertices, and routes correspond to weighted edges representing distance or travel time. This representation was chosen over an adjacency matrix due to the sparse nature of city transport networks, ensuring $O(V+E)$ space efficiency.

% ==================== IMPLEMENTATION ====================
\section{Implementation}

\subsection{Graph Algorithms}

Breadth-First Search (BFS) and Depth-First Search (DFS) are implemented using queue and stack data structures respectively. Dijkstra’s algorithm utilizes a manually implemented min-heap to compute the shortest path between two stations. For the Minimum Spanning Tree (MST), Kruskal's algorithm is implemented using a Union-Find data structure with path compression to optimize cycle detection.

\subsection{Hash Table and Collision Handling}

The vehicle database is implemented using a hash table. To handle the "birthday paradox" and potential clustering, the system employs separate chaining. Each bucket in the table points to a linked list of \texttt{Vehicle} objects.

\subsection{Tree and Heap Modules}

The Binary Search Tree (BST) is utilized for maintaining sorted daily usage trends, allowing for $O(\log n)$ search and insertion. The \texttt{MinHeap} is crucial for the priority-based tasks in the \texttt{Analytics} module, such as assigning the fastest available vehicle to a route based on priority keys.

% ==================== ALGORITHM ANALYSIS ====================
\section{Algorithm Analysis}

\subsection{Graph Algorithm Complexities}

\begin{table}[H]
\centering
\caption{Graph Algorithm Complexities}
\begin{tabular}{lcc}
\toprule
\textbf{Algorithm} & \textbf{Time Complexity} & \textbf{Space Complexity} \\
\midrule
BFS / DFS & $O(V+E)$ & $O(V)$ \\
Dijkstra’s Algorithm & $O((V+E)\log V)$ & $O(V)$ \\
Kruskal's (MST) & $O(E \log E)$ & $O(V+E)$ \\
\bottomrule
\end{tabular}
\end{table}

\subsection{Sorting Algorithm Performance}

The \texttt{SortSearchUtils} module was analyzed to compare the performance of different sorting techniques within the system context.

\begin{table}[H]
\centering
\caption{Sorting Algorithm Analysis}
\begin{tabular}{lccc}
\toprule
\textbf{Algorithm} & \textbf{Best} & \textbf{Average} & \textbf{Worst} \\
\midrule
Bubble Sort & $O(n)$ & $O(n^2)$ & $O(n^2)$ \\
Merge Sort & $O(n \log n)$ & $O(n \log n)$ & $O(n \log n)$ \\
Quick Sort & $O(n \log n)$ & $O(n \log n)$ & $O(n^2)$ \\
Heap Sort & $O(n \log n)$ & $O(n \log n)$ & $O(n \log n)$ \\
\bottomrule
\end{tabular}
\end{table}

% ==================== TESTING ====================
\section{Testing and Results}

Each module was tested independently using unit testing techniques and later integrated into the complete system. Sorting and searching algorithms were validated using multiple datasets ranging from 10 to 1,000 entries. Graph algorithms produced correct traversals, shortest paths, and minimum spanning trees, verified against manual calculations for small sub-graphs. 



% ==================== CONCLUSION ====================
\section{Conclusion and Future Enhancements}

The ITNMS project successfully demonstrates the real-world applicability of core data structures and algorithms. By implementing all data structures manually, the project strengthens conceptual understanding while delivering a functional transportation management system.

Future enhancements include:
\begin{itemize}
    \item \textbf{Persistent Storage:} Implementing File I/O to save the state of the graph and vehicle database.
    \item \textbf{GUI:} Developing a graphical interface using Qt or SFML to visualize the transport network.
    \item \textbf{Real-time Data:} Integrating API support for real-time traffic updates.
\end{itemize}

% ==================== REFERENCES ====================
\begin{thebibliography}{9}

\bibitem{cormen}
T. H. Cormen, C. E. Leiserson, R. L. Rivest, and C. Stein,
\textit{Introduction to Algorithms}, 3rd ed. MIT Press, 2009.

\bibitem{weiss}
M. A. Weiss,
\textit{Data Structures and Algorithm Analysis in C++}, 4th ed. Pearson, 2011.

\bibitem{dijkstra}
E. W. Dijkstra,
“A note on two problems in connexion with graphs,”
\textit{Numerische Mathematik}, vol. 1, pp. 269–271, 1959.

\end{thebibliography}

\end{document}